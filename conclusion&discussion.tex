\chapter{Conclusion and Discussion}\label{chapter:conclusion&discussion}
The Shared Autonomous Mobility (SAM) concept overall has gained a significant positive reputation in many researches and studies, with autonomous shuttles having the potential to mitigate a good number of the existing problems related to traffic, environmental, overall costs incurred, last mile transportation, service fares, etc. that exist with the public transportation today and which have made it unfavourable for an average daily commuter to not opt for a privately owned car. The HEAT (Hamburg Electric Autonomous Transportation) research and development project is a pilot in HafenCity, Hamburg which tests the autonomous shuttle system in an urban environment and aims for the integration of an autonomous minibus (shuttle) service into the regular traffic of HafenCity. With a high focus on safety and operational capabilities, its efficient implementation on HafenCity network demonstrates the positive effects that innovative and ingenious mobility concepts can have in overcoming the limitations of the existing framework.

This thesis focused on the traffic and pedestrian simulation-based analysis through an agent based modeling of the operational requirements of the HEAT autonomous shuttles for a special-event scenario in HafenCity. Taking the scenario of a concert event at the \textit{Elbphilharmonie}, the idea was to check the efficiency and provide solutions for the HEAT autonomous shuttle operation, as a public transport, in providing service to the influx of pedestrians that enter the HafenCity network who, after the event has ended, are coming out from the venue, for a two hour period, i.e., from 18:00 to 20:00 hours. For this purpose, a microsimulation model for the entire HafenCity road network was developed and the public transport services, modeled for autonomous vehicle operation, was included, using the Aimsun Next software. A pedestrian area with one pedestrian entrance (Elbphilharmonie) and three different pedestrian exits were defined in the model through which all the pedestrians would enter and wait for the shuttle service at the PT stop; and finally exit the network after they alight from the vehicle. For this thesis, two primary indicators for measuring the pedestrian demand were selected with respect to Aimsun Next, namely, the \textit{Pedestrian Waiting Time at PT Stop} and the \textit{Pedestrian Travel Time}.

Keeping in mind the objectives of this study, five main scenarios were drawn and evaluated based on the two parameters for the implementation of methodology. The results were tabulated and plots were drawn for each parameter values in all the scenarios. The first two scenarios dealt with the initial existing cases of situation when the PT operates with time interval between departures at 10 minutes (\ref{scenario1}) and 5 minutes (\ref{scenario2}).The unreasonable and incredibly high average values of waiting time at PT stop (17.97 minutes and 15.14 minutes, respectively) and overall travel time (38.86 minutes and 23.39 minutes, respectively) values showed the insufficiency in the current service for the pedestrian demand. From careful examination of the results and plots, a \textit{peak hour} period was identified for which the cumulative increase in the demand values were impractical, which was found to be from 19:00 to 20:00 hours. This brought the focus on looking at this as a dynamic scheduling at the peak hour problem and attempting to solve for this through the implementation of certain functions from the Aimsun Next API module through direct programming (in this study, Python). Functions for dynamically changing mainly three different properties or factors influencing the public transport operation were investigated. While there was a direct function available for changing the maximum desired speed (from 15 km/h to 25 km/h) and modifying the PT vehicle route in such a way that the PT stop which was not utilized by the pedestrians to board or alight from the bus is skipped (the \textit{"Am Sandtorkai"} bus stop in the network) during the peak hour (\ref{scenario3}), a direct function could not be found for changing/influencing the schedule or timetable of the PT vehicle route. However, a workaround for solving for this issue was found which involved a case when the vehicle operates on the route with the time interval between departures at 5 minutes as usual and during the peak hour, an additional vehicle is introduced in the network using Aimsun API module with time interval between departures at every 3 minutes on the route. (\ref{scenario4}).

For the case of \ref{scenario3}, the results showed a considerable but not sufficient decrease in the pedestrian waiting time (11.17 minutes) and the overall pedestrian travel time (18.76 minutes). The relatively improved results also showcased that the approach of implementing API functions during the peak hour could be a considerable one for this study. This became evident when the results for \ref{scenario4} were examined. When compared to existing scenarios (\ref{scenario1} and \ref{scenario2}), it was observed that the pedestrian demand decreases by a large extent with the pedestrian waiting time and the overall travel time values (7.67 minutes and 15.84 minutes, respectively) reducing by more than 50\%. The final scenario (\ref{scenario5}) considered the case of combining the scenarios 3 and 4 (in sections \ref{scenario3} and \ref{scenario4} respectively). The results and their plots for the pedestrian waiting times (5.25 minutes) and overall travel times (11.98 minutes) show, upon proper implementation, a compelling use case of the HEAT autonomous shuttle service as a user-oriented public transportation even in a worst case (special-event) scenario of pedestrian demand peaks the network of HafenCity, such as the one considered in this thesis study, and beyond. A possible direction for further research and related studies in the future could be conducting an operational fleet management study of deployment of autonomous shuttles which is based on the pedestrian requests making it a truly demand-responsive request based system.

Therefore, it can also be understood that a carefully planned and efficiently managed fleet of the user-oriented shared autonomous vehicles (such as the HEAT autonomous shuttles), when shifted from the pilot stage and brought to the urban areas as a feeder service with proper implementation depending on the urban environments and population, have the potential to bring reforms and ameliorate the public transportation systems of the future along with shaping the way urban mobility is seen today.


