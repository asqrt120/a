\chapter{\abstractname}
In recent decades, economic trends driven by globalization have both fueled economic growth and exacerbated income disparities across nations and within societies. This paper critically examines the complex relationship between globalization and economic inequality, focusing on how global economic integration has contributed to rising disparities in wealth distribution. The study first analyzes the theoretical foundations of economic inequality, followed by an exploration of key drivers linked to globalization, such as trade liberalization, technological advancements, and capital flows. Using case studies and empirical data, this research highlights both the positive and negative impacts of these trends, with particular emphasis on developing versus developed economies. The analysis reveals that while globalization has facilitated significant economic opportunities, its benefits have been unevenly distributed, leading to pronounced economic polarization and social tensions. Additionally, the paper discusses potential policy interventions to mitigate inequality, exploring strategies such as progressive taxation, improved social safety nets, and inclusive growth initiatives. The findings underline the need for a balanced approach to globalization—one that promotes equitable economic outcomes while fostering global cooperation. Ultimately, this study provides a comprehensive overview of the ongoing challenges and future prospects in addressing economic inequality in an increasingly interconnected world.


% The idea from evaluating the operations of the HEAT project in special scenarios is to understand how the use of self-driving (autonomous) vehicles in the public transportation sector could not only make a positive impact on the daily activities of the users, but also, if implemented properly by the cities and fleet managers, could essentially change or shape the designing of the future control center workplaces in the long term.



% \makeatletter
% \ifthenelse{\pdf@strcmp{\languagename}{english}=0}
% {\renewcommand{\abstractname}{Kurzfassung}}
% {\renewcommand{\abstractname}{Abstract}}
% \makeatother

% %\chapter{\abstractname}

% %TODO: Abstract in other language
% %\begin{otherlanguage}{ngerman} % TODO: select other language, either ngerman or english !

% %\end{otherlanguage}


% % Undo the name switch
% \makeatletter
% \ifthenelse{\pdf@strcmp{\languagename}{english}=0}
% {\renewcommand{\abstractname}{Abstract}}
% {\renewcommand{\abstractname}{Kurzfassung}}
% \makeatother

% \cleardoublepage{}