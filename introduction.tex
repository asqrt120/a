\chapter{Introduction}\label{chapter:introduction}

Global economic trends have become a central focus of academic and policy discussions in recent decades, driven by rapid globalization, technological advancements, and shifting geopolitical landscapes. Among these trends, the relationship between globalization and economic inequality stands out as a complex and multifaceted issue that continues to evolve. While globalization has undoubtedly fueled economic growth and interconnected markets worldwide, it has also been associated with widening income disparities, both within and between nations. Understanding this duality is crucial for developing effective policies that can balance economic prosperity with social equity.

The central research question of this paper is: How has the globalization trend influenced economic inequality in developed and developing countries over the past three decades? Understanding the dynamics of globalization’s impact on inequality is crucial for assessing the effectiveness of economic policies and ensuring more equitable development outcomes in both high-income and low-income regions. Developed countries, with their advanced economies, have often reaped substantial benefits from globalization, but they have also witnessed growing income inequality. On the other hand, many developing nations have experienced mixed outcomes, where globalization has brought both opportunities and challenges, contributing to a complex pattern of wealth distribution.

The aim of this paper is to critically examine the relationship between globalization and economic inequality by analyzing trends in both developed and developing countries. Through a detailed literature review and trend analysis, the paper will explore how globalization has differentially impacted these two groups. A series of case studies will provide specific examples, highlighting the nuances of these trends and allowing for a comparative analysis. Ultimately, the paper seeks to answer whether globalization has exacerbated or mitigated economic inequality and to what extent regional and national factors play a role in this dynamic.

The paper is structured as follows: First, the theoretical background in chapter \ref{chapter:theoreticalBackground} will be discussed to provide a foundation for understanding key concepts such as globalization and economic inequality. Following this, a review of existing literature will be presented in chapter \ref{chapter:literatureReview}, focusing on the different perspectives regarding the globalization-inequality relationship. The core analysis in chapter \ref{chapter:trendsAnalysis} will then examine global trends, distinguishing between the experiences of developed and developing countries, supported by specific case studies. The discussion section will critically evaluate these findings, drawing connections between theory and observed trends, before concluding with key insights and an outlook on future research directions... \ref{chapter:conclusion&discussion}.


% This paper explores the intricate interplay between globalization and economic inequality, focusing specifically on how the integration of global markets has influenced wealth distribution. The research seeks to answer the question: How has globalization contributed to economic inequality, and what are the potential implications for future economic stability? By critically examining existing literature and employing relevant economic theories, this study aims to provide a comprehensive analysis of this global trend and its broader implications.
% The paper is structured as follows: First, the theoretical background will be discussed, providing a foundation for understanding key concepts such as globalization, economic inequality, and relevant economic models. The subsequent section delves into the trend development using case studies and empirical data, highlighting how globalization has impacted different regions and socio-economic groups. Finally, the discussion will analyze the potential socio-economic consequences of these trends, considering both positive and negative outcomes. The paper concludes with a summary of the findings, offering insights into future research directions and potential policy interventions.

% \section{Background and Motivation}
% Transportation systems are primitive to the modern society and serve as a contributing factor in the socio-economic activities. At an individual level, they shape the citizens' mobility needs for their daily activities. However, progress is a way of life and just like for any other scientific field, it also holds true for the mobility sector. Although, there has been many a considerable amount of headway made for the development of this sector, technological advantages and numerous commitments, the EU's transportation sector saw a significant rise in the emissions in the previous decade, which was more than the 1990 level, with road transport contributing to about 21\% of carbon dioxide emissions of the EU \citep{2017EUcommissionreport}. Adding to this, reasons such as pollution, traffic congestions, rising fuel prices, etc. do not paint a good picture of the traditional cars today and it becomes rather impractical to have a city where every individual drives a privately owned car.

\section{Paper Structure}
% An overall approach adopted for this research and study; and the results concluded are presented in the rest of this thesis. Chapter \ref{chapter:Literature Review} reviews the literature on the autonomous vehicle technology and its impact on the world today, the shared autonomous mobility as an extension of carsharing and shared economy concepts and fleet management and dynamic scheduling. While covering the literatures on the famous vehicle routing problem, it also provides the interesting work done on fleet management in case of the shared autonomous vehicles, and finally highlighting the literature gaps at the end. Chapter \ref{chapter:Methodology} formulates the methodology for this study. After a brief introduction to the study area and the HEAT project, it describes the general methodology adopted for the thesis. The simulation setup with Aimsun Next is described in detail and the Aimsun Next API architecture, its principles, use, applications in the microsimulation framework are discussed. An investigation for the suitable API microsimulation functions are done for the purpose of dynamically changing certain factors that influence the public transport operation, which could be used to approach the thesis objectives in the methodology. The chapter ends with summarizing the methodology adopted in the thesis with the help of a graphical representation. Chapter \ref{chapter:results} provides all the results obtained for five scenarios drawn with different cases of the HEAT autonomous shuttle operation with respect to two determining indicators of this thesis. Finally, in chapter \ref{chapter:conclusion&discussion} includes conclusion and interpretation of the results obtained along with a discussion on possible directions for further research in the future.